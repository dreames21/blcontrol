\chapter{Data Analysis}



\section{Optic Resolution\label{sec:res}}

\subsection{Half-Power Diameter\label{sec:hpd}}

The optic's half-power diameter (HPD) is found by fitting the count data to
the following function, which is the integral of a gaussian:
\begin{equation}
  f(d) = A(1-\exp{\left(-\frac{(d-d_0)^2}{4b}\right)} + f_0
    \label{eq:hpd_fit}
\end{equation}
where $d$ is the diameter of the pinhole, $f(d)$ is the count rate at
that diameter, and $A$, $d_0$, $b$, and $f_0$ are fitting
parameters. The HPD is then given by the value of $d$ where $f(d)$ is
equal to half of its maximum value:
\begin{equation}
  \label{eq:hpd}
  d_{1/2} = d_0 + 2 \sqrt{b \log{ \left( \frac{2A}{f_0 + A} \right)}}
\end{equation}

A script to perform this fit and produce an accompanying plot is found at
\path{blcontrol/scripts/HPD.py}. To run the script
\begin{verbatim}

\end{verbatim}


\subsection{Resolution\label{sec:res}}

To calculate the spatial resolution from the HPD, we must account for the
magnificaiton factor of the optic and the finite size of the source.  We assume
that the source spot size and the optic spatial resolution add in quadrature to
contribute to the measure HPD. Therefore, if the source size is $d_\text{src}$
and the magnification factor is $M$, then the optic's spatial resolution $r$ can
be approximated by
\begin{equation}
  \label{eq:sp_res}
  r = \sqrt{ \left( \frac{d_{1/2}}{M} \right)^2 - d_\text{src}^2 }
\end{equation}
where $d_{1/2}$ is the measured HPD from Sec.~\ref{sec:hpd}.

Given the spatial resolution, the angular resolution $\theta$ of the optic is
easily estimated using the source-to-optic distance, $u$:
\begin{equation}
  \label{eq:ang_res}
  \theta = 2 \arctan{ \left( \frac{r}{2u} \right) }
\end{equation}

\section{Reflectivity}

The reflectivity of the optic is given by the ratio of the count rate incident
on the optic in to the count rate focused by the optic onto the detector, for a
given energy. The output spectrum is measured directly during the data
acquisition (see Sec.~\ref{sec:spec-collect}). However, given the geometry of the
beamline and the size of the detector, we can't directly measure the spectrum
incident on the optic, but we can estimate it by measuring the source output
with the detector placed at the focal length of the optic. Then, two correction
factors must be applied. First, we must account for the difference in collecting
area between the detector and the optic. Then, a correction must be applied to
account for the fact that the intensity drops off with distance as
$1/r^2$. Putting all this together, the reflectivity of the optic as a function
of energy, $R(E)$, is given by:
\begin{equation}
  \label{eq:refl}
  R(E) = \left( \frac{ I_\text{optic} }{ I_\text{source} } \right)
  \left( \frac{ A_\text{optic} }{ A_\text{detector} } \right)
  \left( \frac{ u }{ u+v } \right)^2
\end{equation}

A script to perform this computation can be found at
\path{blcontrol/scripts/reflectivity.py}. 
