\chapter{Data Analysis}



\section{Optic Resolution\label{sec:res}}

\subsection{Half-Power Diameter\label{sec:hpd}}

The optic's half-power diameter (HPD) is found by fitting the count data to
the following function, which is the integral of a gaussian:
\begin{equation}
  f(d) = A(1-\exp{\left(-\frac{(d-d_0)^2}{4b}\right)} + f_0
    \label{eq:hpd_fit}
\end{equation}
where $d$ is the diameter of the pinhole, $f(d)$ is the count rate at
that diameter, and $A$, $d_0$, $b$, and $f_0$ are fitting
parameters. The HPD is then given by the value of $d$ where $f(d)$ is
equal to half of its maximum value:
\begin{equation}
  \label{eq:hpd}
  d_{1/2} = d_0 + 2 \sqrt{b \log{ \left( \frac{2A}{f_0 + A} \right)}}
\end{equation}

A script to perform this fit and produce an accompanying plot is found at
\path{blcontrol/scripts/HPD.py}. To run the script
\begin{verbatim}

\end{verbatim}


\subsection{Resolution}

To calculate the spatial resolution from the HPD, we must account for the
magnificaiton factor of the optic and the finite size of the source.  We assume
that the source spot size and the optic spatial resolution add in quadrature to
contribute to the measure HPD. Therefore, if the source size is $d_\text{src}$
and the magnification factor is $M$, then the optic's spatial resolution $r$ can
be approximated by
\begin{equation}
  \label{eq:sp_res}
  r = \sqrt{ \left( \frac{d_{1/2}}{M} \right)^2 - d_\text{src}^2 }
\end{equation}
where $d_{1/2}$ is the measured HPD from Sec.~\ref{sec:hpd}.

Given the spatial resolution, the angular resolution $\theta$ of the optic is
easily estimated using the source-to-optic distance, $u$:
\begin{equation}
  \label{eq:ang_res}
  \theta = 2 \arctan{ \left( \frac{r}{2u} \right) }
\end{equation}

\section{Reflectivity}

On the most basic level, the reflectivity of the optic at a given energy is
simply the ratio of the number of photons reflected from the optic to the number
of photons incident on the optic. This is called the \textit{double-bounce
  reflectivity} because the photons are reflected once from each of the two ends
of the optic. The \textit{single-bounce reflectivity} is the reflectivity after
a single reflection from the optic, and corresponds to the reflectivity of the
multilayer coating. If we assume that the multilayer is exactly the same on both
ends of the optic, and thus they have the same single-bounce reflectivity, then
we have that $R_\text{double-bounce} = R_\text{single-bounce}^2$. Of course, the
multilayer will not be exactly the same on each end of the optic, but it's a
useful approximation so that we can compare the optic to multilayers on flats
and other samples.

When we take the long optic spectrum after aligning the optic, we are directly
measuring the number of photons reflected from the optic. The input to the optic
is related to, but not the same as, the spectrum we take from the source after
removing the optic. It is not the same because the collecting area of the optic
is different from the collecting area of the detector, and because the detector
is farther from the source than the optic is, and the flux drops off as
$1/r^2$. Taking these into account, the input spectrum $I_\text{input}(E)$ is
related to the source spectrum taken in Sec.~\ref{sec:spec-collect},
$I_\text{src}(E)$, by
\begin{equation}
  \label{eq:src-2-input}
  I_\text{input}(E) = \left( \frac{ A_\text{optic} }{ A_\text{detector} }
  \right) \left( \frac{ u+v }{ u } \right)^2 I_\text{src}(E),
\end{equation}
where $A_\text{optic}$ and $A_\text{detector}$ are the respective collecting
areas of the optic and detector, $u$ is the distance from source to optic, and
$v$ is the distance from optic to detector. Then taking the ratio of
$I_\text{optic}$, the optic output spectrum, to $I_\text{input}$, yields the
double-bounce reflectivity, and taking the square root of this yields the
single-bounce reflectivity.

A Python script to perform this computation, \path{reflectivity.py}, is provided
in \path{blcontrol/scripts}. The most convenient way to run the script is by
copying it to the directory where the optic and source spectrum data files are
located. The variables at the beginning of the script will need to be edited for
the parameters used, and then it can be run from the terminal using
\path{./reflectivity.py}. The script produces a text file and a plot of the
calculated reflectivity in each of the MCA channels.

%%% Local Variables:
%%% mode: latex
%%% TeX-master: "Beamline_Manual"
%%% End:
