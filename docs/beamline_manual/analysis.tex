\chapter{Data Analysis}



\section{Optic Resolution\label{sec:res}}

\subsection{Half-Power Diameter\label{sec:hpd}}

The optic's half-power diameter (HPD) is found by fitting the count data to
the following function, which is the integral of a gaussian:
\begin{equation}
  f(d) = A(1-\exp{\left(-\frac{(d-d_0)^2}{4b}\right)} + f_0
    \label{eq:hpd_fit}
\end{equation}
where $d$ is the diameter of the pinhole, $f(d)$ is the count rate at
that diameter, and $A$, $d_0$, $b$, and $f_0$ are fitting
parameters. The HPD is then given by the value of $d$ where $f(d)$ is
equal to half of its maximum value:
\begin{equation}
  \label{eq:hpd}
  d_{1/2} = d_0 + 2 \sqrt{b \log{ \left( \frac{2A}{f_0 + A} \right)}}
\end{equation}

A script to perform this fit and produce an accompanying plot is found at
\path{blcontrol/scripts/HPD.py}. The \path{find_HPD} function defined in the
script takes two lists as input, a list of the pinhole diameters, and a list of
the corresponding number of counts recorded in the ROI using that pinhole. It
will fit a function like \label{eq:hpd_fit} and produce a plot that shows the
fitted function and the calculated HPD. You can use this function by importing
it and running it from the Python interpreter in a terminal.


\subsection{Resolution}

To calculate the spatial resolution from the HPD, we must account for the
magnificaiton factor of the optic and the finite size of the source.  We assume
that the source spot size and the optic spatial resolution add in quadrature to
contribute to the measure HPD. Therefore, if the source size is $d_\text{src}$
and the magnification factor is $M$, then the optic's spatial resolution $r$ can
be approximated by
\begin{equation}
  \label{eq:sp_res}
  r = \sqrt{ \left( \frac{d_{1/2}}{M} \right)^2 - d_\text{src}^2 }
\end{equation}
where $d_{1/2}$ is the measured HPD from Sec.~\ref{sec:hpd}.

Given the spatial resolution, the angular resolution $\theta$ of the optic is
easily estimated using the source-to-optic distance, $u$:
\begin{equation}
  \label{eq:ang_res}
  \theta = 2 \arctan{ \left( \frac{r}{2u} \right) }
\end{equation}

\section{Reflectivity}

\subsection{Calculation}

On the most basic level, the reflectivity of the optic at a given energy is
simply the ratio of the number of photons reflected from the optic to the number
of photons incident on the optic. This is called the \textit{double-bounce
  reflectivity} because the photons are reflected once from each of the two ends
of the optic. The \textit{single-bounce reflectivity} is the reflectivity after
a single reflection from the optic, and corresponds to the reflectivity of the
multilayer coating. If we assume that the multilayer is exactly the same on both
ends of the optic, and thus they have the same single-bounce reflectivity, then
we have that $R_\text{double-bounce} = R_\text{single-bounce}^2$. Of course, the
multilayer will not be exactly the same on each end of the optic, but it's a
useful approximation so that we can compare the optic to multilayers on flats
and other samples.

When we take the long optic spectrum after aligning the optic, we are directly
measuring the number of photons reflected from the optic. The input to the optic
is related to, but not the same as, the spectrum we take from the source after
removing the optic. It is not the same because the collecting area of the optic
is different from the collecting area of the detector, and because the detector
is farther from the source than the optic is, and the flux drops off as
$1/r^2$. Taking these into account, the input spectrum $I_\text{input}(E)$ is
related to the source spectrum taken in Sec.~\ref{sec:spec-collect},
$I_\text{src}(E)$, by
\begin{equation}
  \label{eq:src-2-input}
  I_\text{input}(E) = \left( \frac{ A_\text{optic} }{ A_\text{detector} }
  \right) \left( \frac{ u+v }{ u } \right)^2 I_\text{src}(E),
\end{equation}
where $A_\text{optic}$ and $A_\text{detector}$ are the respective collecting
areas of the optic and detector, $u$ is the distance from source to optic, and
$v$ is the distance from optic to detector. Then taking the ratio of
$I_\text{optic}$, the optic output spectrum, to $I_\text{input}$, yields the
double-bounce reflectivity, and taking the square root of this yields the
single-bounce reflectivity.

A Python script to perform this computation, \path{reflectivity.py}, is provided
in \path{blcontrol/scripts}. The most convenient way to run the script is by
copying it to the directory where the optic and source spectrum data files are
located. The variables at the beginning of the script will need to be edited for
the parameters used, and then it can be run from the terminal using
\path{./reflectivity.py}. The script produces a text file and a plot of the
calculated reflectivity in each of the MCA channels.

\subsection{Modeling}

To model the reflectivity of the optic, we consider the path of a ray emanating
from the source, reflecting from each end of the optic, and impinging on the
detector. It is characterized by energy $E$ and the angle $\phi$ that it makes
with respect to the axis of the optic. We assume that the intensity of the
source is constant in $\phi$ so that $I(\phi, E) = I_0(E)$ where $I_0(E)$ is
simply the spectrum of the source. The ray reflects from one end of the optic
(we'll assume it's the $h$-end, but it's the same in either orientation) at a
graze angle $\theta_h$ which is a function of $\phi$:
$\theta_h = \theta_h(\phi)$. The multilayer on this end has a reflectivity as a
function of energy and graze angle $R'_h(\theta_h, E) = R_h(\phi, E)$ that is
determined by the properties of the multilayer, including the $d$-spacing,
$\Gamma$, and microroughness. After reflecting from the $h$ end, the intensity
$I'(\phi, E) = R_h(\phi, E) I_0(E)$. The ray then impinges on the $e$-end of the
optic which has a reflectivity function $R_e(\phi, E)$, and the intensity
$I''(\phi, E) = R_h(\phi, E) R_e(\phi, E) I_0(E)$. The ray then continues to the
detector. Thus if we divide through by the source spectrum, the two-bounce
reflectivity of the optic as a whole is

\begin{equation}
  R_\text{opt}(\phi, E) = R_h(\phi, E) R_e(\phi, E) .
  \label{eq:dbl_bounce}
\end{equation}

Of course, the detector is not sensitive to the angle of incidence of the
incoming rays, only the energy. So we integrate over all values of $\phi$ which
produce a valid ray that will reflect from the optic. These limits of
integration $\phi_1$ and $\phi_2$ can be determined from the design geometry of
the optic. The area element $dA$ in this case is the area of the ring on the $h$-end
of the optic where the rays with angles between $\phi$ and $\phi + d\phi$ will
reflect. Then divide by the total area of the $h$ end. All together, this looks
like:
\begin{equation}
  \label{eq:area_int}
  R_\text{opt}(E) = \frac{\int^{\phi_1}_{\phi_2} R_h(\phi, E) R_e(\phi, E) dA}{
    \int^{\phi_1}_{\phi_2} dA } .
\end{equation}

In practice, actually calculating $dA$ and doing the integral is not worth the
hassle. We can get a pretty good approximation simply by averaging over
$\phi$. We would also like to use the actual graze angles $\theta_h$ and
$\theta_e$ as variables, as this is what we will need to input into IMD.  We
will integrate from the edge to the intersection, keeping in mind that a ray
that reflects near the edge on one end will reflect near the edge on the other
end, and similarly for a ray near the intersection. Thus,
\begin{equation}
  \label{eq:phi_int}
  R_\text{opt}(E) =\frac{1} {
    \int^{\phi_\text{edge}}_{\phi_\text{int}}  d\phi } \int^{ \phi_\text{edge}
  }_{ \phi_\text{int}}
  R_h(\theta_h, E) R_e( \theta_e, E) d\phi .
\end{equation}
Equation~\ref{eq:phi_int} is used as the basis for the numerical model that is
computed using IMD and Python. IMD is used to calculate $R_h$ and $R_e$, and a
Python script performs the integration.
% The Python script \path{blcontrol/scripts/model.py} gives an example of how to
% produce a model for the reflectivity of a Wolter optic, using IMD \path{.sav}
% files as starting points for the model. IMD is used to calculate the
% reflectivity functions for the multilayers on each end of the optic as functions
% of energy and graze angle. The Python script combines these reflectivity
% functions and averages over them to produce a model of optic reflectivity as a
% function of energy.

To calculate the reflectivity in IMD, first define the multilayer stack for the
$h$-end of the optic, with an appropriate $d$-spacing, $\Gamma$, and
microroughness. Also include the Pt and NiCo substrate if necessary. Then, under
Independent Variables in the IMD window, define the range of energies and graze
angles. The graze angles for the $h$-end are defined in the mandrel design
parameters, and we proceed from the edge to the intersection. The number of
energy values you use should be comparable to the number of MCA channels used to
collect the data, and $~20$ graze angle values is usually sufficient. Calculate
the reflectivity in the menu Calculate $\rightarrow$ Specular Optical Functions
and Fields, and then save the \path{.imd} file. Close the plot and export to a
\path{.sav} file from the menu using File $\rightarrow$ Export Results to IDL
Save File.... Repeat this process for the $e$-end of the optic, using the same
energy values and the same number of graze angles, but using the values defined
for the $e$-end in the mandrel design. When finished, you should have a
\path{.imd} and a \path{.sav} each for the $h$ end and the $e$ end of the optic
which represent their respective reflectivity functions.

The Python script to perform the integration is found at
\path{blcontrol/scripts/model.py}. It parses the \path{.sav} files, performs the
multiplication of the reflectivity functions, and averages over the graze
angles. The script also uses a Gaussian smoothing filter to simulate the finite
energy resolution of the detector. It then plots the square root of
$R_\text{opt}$, which is the single-bounce reflectivity, along with the measured
single-bounce reflectivity for comparison. You can use this plot to go back and
tweak various model parameters such as the $d$-spacing, microroughness, and
density of each end of the optic, and the energy resolution of the detector, to
see how they affect the model.


%%% Local Variables:
%%% mode: latex
%%% TeX-master: "Beamline_Manual"
%%% End:
