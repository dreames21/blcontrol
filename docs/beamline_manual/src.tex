\chapter{X-ray Source}

This chapter discusses the operation of the Jupiter 5000 X-ray tube
sources. Section~\ref{sec:src_startup} covers the procedure for starting up the
source under normal conditions. If the source is new, or hasn't been used in
more than three months, then it must be started up using the conditioning
procedure outlined in Section~\ref{sec:conditioning}. Finally,
Section~\ref{sec:shutdown} discusses how to safely shut down the X-ray source.

%add citations to source materials from oxford and spellman (generator)

\section{Normal startup procedure\label{sec:src_startup}}

\begin{enumerate}

\item Mount the x-ray tube to the aluminum stand with the copper
  collimator. Affix the thermocouple wire to the outside of the x-ray tube, near
  the center, with tape. Start up cooling fans by plugging the 12V power supply
  into the power strip.

\item Ensure that the on/off switch on the x-ray power supply is in the off
  position. Connect the blue filament cable to the BNC input on the x-ray
  tube. Insert the high voltage cable into the x-ray tube and screw down the
  connector. Connect the ground wire to the screw next to the high-voltage input
  and screw down snugly.

\item Double check that all shielding is in place, including copper pipes, brass
  optic housing, and leaded glass at the detector end of the beam. Use brass
  foil to cover any gaps in the shielding. Also make sure everyone in the room
  is wearing proper dosimetry.

\item Turn on the power supply by moving the switch to the on position. The
  front panel should read \textbf{0.0 mA, 0.0 kV}, and the “HV OFF” button
  should light up.

\item Press and hold the HV OFF button to display the preset current and
  voltage. When the HV is enabled, it will power up to this preset value. The
  value can be adjusted by turning the mA and kV knobs while holding down the HV
  OFF button. Set the preset current to 0.0 mA and the preset voltage to 10.0
  kV. This is the minimum voltage rating for the Jupiter 5000 x-ray tubes. Note
  that no actual output is being produced at this stage.

\item \label{item:on} Press the HV ON button to power up the supply to the
  preset voltage. The HV ON button should illuminate and the front panel should
  read 0.0 mA, 10.0 kV. Using the Geiger counter, check the beamline area for
  radiation leakage, especially in places where two pieces of shielding come
  together (such as the holes in the brass optic housing). If any radiation
  leakage is detected, or any warning lights illuminate on the front panel of
  the power supply, turn off the output by pressing HV OFF. Record the time and
  tube temperature reading in the logbook.

\item Increase the output voltage to \textbf{20.0 kV}. Sweep the beamline area
  again using the Geiger counter. Record the time and tube temperature.

\item Increase the output current to \textbf{0.025 mA}. Sweep the beamline area
  again using the Geiger counter.  Record the time and tube temperature. Wait 30
  seconds after increasing the current before proceeding to the next step.

\item Increase to the desired voltage in increments of 10 kV, checking for
  radiation leakage, recording the time and temperature, and waiting 30 seconds
  between each increase. Do NOT exceed 50 kV.  Lower the voltage and/or the
  current if the temperature nears 45C. Do NOT allow the temperature to exceed
  49.5C.

\item Once the desired voltage is reached, test the flux by taking a 30 second
  test spectrum with the detector. If more flux is needed, the current may be
  increased in increments of 0.025 mA. Check for radiation leakage, record the
  time and tube temperature, and wait 30 seconds between each increase.  If the
  detector dead time reaches 5\% or higher or the temperature rises too much,
  decrease the current. Do NOT exceed 1 mA.  You should generally not need to
  exceed 0.1 mA to obtain sufficient flux.\label{item:finish}

\item The x-ray source is now powered up and ready for the alignment process to
  begin. While the source is powered on, do not reach onto or over the
  benchtop. Every 15 minutes or so, record the temperature of the tube, make
  sure the current and voltage is stable, check for error messages on the front
  panel of the power supply, and sweep the beamline area for radiation leakage.

\end{enumerate}

\section{Tube conditioning procedure\label{sec:conditioning}}

This procedure must be used when starting up a new tube or a tube that has been
in storage for longer than three months. During a period of storage, residual
gasses can accumulate in the x-ray tube vacuum, and applying a high voltage
without first conditioning the tube can cause arcing, which can lead to
permanent damage to the tube~\cite{tube_conditioning}.

\begin{enumerate}

\item Follow steps from Section~\ref{sec:src_startup} up to step~\ref{item:on}
  to start up the tube at \textbf{0.0 mA, 10.0 kV} and check for radiation
  leakage.

\item If any instability is noted on the mA meter, allow it to stabilize to
  display 0 mA. Operate at this condition for at least 15 minutes.

\item Increase the beam current to \textbf{0.2 mA} and sweep the beamline area
  for radiation leakage with the Geiger counter. Note the tube temperature and
  any instability in the mA meter. Maintain this setting for 5 minutes or
  longer, until no instability is noted on the mA meter.

\item Increase high voltage in 5 kV steps at 5 minute intervals until \textbf{25
  kV} is reached. After each increase, note the tube temperature and check for
  radiation leakage. Hold 5 minutes at these conditions.

\textbf{Note:} if any instability (especially loud popping) is observed, lower
the kV setting to the previous step. Allow mA to stabilize at least 5 minutes
before increasing settings again.

\item Increase beam current to maximum current rating, \textbf{1.0 mA}. Note the
  temperature, current stability and radiation leakage. Hold for 5 minutes.

\item Continue to increase high voltage as in step 4, in 5 kV steps every 5
  minutes until the maximum rated voltage, \textbf{50 kV}, is reached. Note the
  tube temperature and check for radiation leakage and instability after each
  increase. Do not allow the tube temperature to exceed 49.5C. Allow at least 5
  minutes at full power to insure that the tube is operating correctly.

\item Slowly reduce the current to 0.025 mA, then slowly reduce the voltage to
  the desired operating voltage. Note the tube temperature and check for
  radiation leakage. Proceed with step~\ref{item:finish} onward from
  Section~\ref{sec:src_startup}.


\end{enumerate}

\section{Shutdown procedure\label{sec:shutdown}}

\begin{enumerate}

\item Slowly dial current down to \textbf{0.0 mA}. Wait 10 seconds.

\item Slowly dial voltage down to \textbf{10.0 kV}. Wait 10 seconds.

\item Press the HV OFF button. Wait until the current and voltage readings are
  stable at \textbf{0.0 mA, 0.0 kV}.

\item Once current and voltage are stable at zero, turn the power switch on the
  power supply to the off position.

\item Allow tube temperature to cool to room temperature, then unplug 12V power
  supply to disable cooling fans.

\end{enumerate}
