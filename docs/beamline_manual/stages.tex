\chapter{Stages}

\section{Overview}

The stages used in this beamline are from Zaber Technologies, Inc., and have
built-in drivers, controllers, and encoders. They can be controlled via software
or by turning the knobs located on the side of the motor housing. In general
they need very little setup before use, and many of the relevant information has
already been covered in Chapter~\ref{sec:software}. The following sections will
cover the procedure for setting up a new stage in the beamline.

\section{Setting up a new stage}

\subsection{Setting to correct protocol}

The first step to using a brand new Zaber stage is to ensure that the
communication protocol and baud rate are set properly to communicate with the
\path{blcontrol} software and with each other. There are two communication
protocols for the Zaber stages, Binary and ASCII. The software communicates
using the binary protocol because some of the stages aren't capable of using the
ASCII protocol, and we use a baud rate of 9600 which is the default for the
binary protocol.

The easiest way to change the communication protocol is by using the Zaber
Console software, which runs on Windows only. It can be installed from the Zaber
website at \url{https://www.zaber.com/wiki/Software/Zaber_Console}. Once it is
installed, connect the stage to the PC. You may have to use a serial connector
instead of USB if the Windows PC doesn't have the correct drivers for a
USB-to-RS232 converter. Open the software, open the serial port, and press the
``Find Devices'' button. Choose the option to convert all devices to Binary
protocol at 9600 baud. Once this process is complete, you can disconnect the
stage from the Windows PC and install it in the beamline.  

\subsection{Editing configuration file}

The structure of the configuration files is discussed in detail in
Sec.~\ref{sec:config}. For a new stage, it is necessary to define a few
parameters in the configuration file before it can be used with the
software. Give the motor a short name under ``Motor Names,'' and then this name
will be used to set the configuration values in the following
sections. Following the instructions from Sec.~\ref{sec:config}, create entries
for ``Motor Res,'' ``Max Current,'' and ``Travel,'' following the specification
sheet for the new device. Save the configuration file, and you are ready to
start up the software and use the new stage. Note that the new stage, and any
stages that were unplugged from power during the installation of the new stage,
will likely have to be homed before use. See Sec.~\ref{sec:motor_control} for
details on homing the stages.

\subsection{Editing stage firmware settings}

A list of all the settings that can be edited in the stage firmware can be found
in the Zaber Binary Protocol Manual\cite{zaber_bin_protocol}. Most of the
settings will not need to be changed, but you may want to edit the running
and/or holding currents if the motors seem to be getting hot. These can be
changed in the Zaber Console software on Windows.

%%% Local Variables:
%%% mode: latex
%%% TeX-master: "Beamline_Manual"
%%% End:
